\documentclass[12pt]{article}
\usepackage{listings}
\usepackage{color}
\textwidth=7in
\textheight=9.5in
\topmargin=-1in
\headheight=0in
\headsep=.5in
\hoffset  -.85in

\definecolor{mygray}{rgb}{0.4,0.4,0.4}
\definecolor{mygreen}{rgb}{0,0.8,0.6}
\definecolor{myorange}{rgb}{1.0,0.4,0}

\lstset{
basicstyle = \ttfamily,columns=fullflexible,
commentstyle=\color{mygray},
frame=single,
numbers=left,
numbersep=5pt,
numberstyle=\tiny\color{mygray},
keywordstyle=\color{mygreen},
showspaces=false,
showstringspaces=false,
stringstyle=\color{myorange},
tabsize=2
}

\pagestyle{empty}

\renewcommand{\thefootnote}{\fnsymbol{footnote}}

\begin{document}

\begin{center}
{\bf lab 20 Priority Queues}

\end{center}

\setlength{\unitlength}{1in}

\begin{picture}(6,.1) 
\put(0,0) {\line(1,0){6.25}}         
\end{picture}

\renewcommand{\arraystretch}{2}
\setlength{\tabcolsep}{6pt} % General space between cols (6pt standard)
\renewcommand{\arraystretch}{.5} % General space between rows (1 standard)

\vskip.15in
\noindent\textbf{Instructions:} In this lab implement a priority queue using something better than O(n) for add and remove.

Implement the following interface:
\begin{lstlisting}[language=C++]{Name=test2}
#ifndef PRIORITY_QUEUE_H
#define PRIORITY_QUEUE_H

template<class T>
class PriorityQueue {
    private:
        /* Class to implement.*/
    public:
        /* Empty constructor shall create an empty PriorityQueue! */
        PriorityQueue();

        /* Do a deep copy of queue into the this.
         * Note: This one uses a reference to a PriorityQueue!
         */
        PriorityQueue(const PriorityQueue<T> &pq);

        /* Deconstructor shall free up memory */
        ~PriorityQueue();

        /* Return the current length (number of items) in the queue */
        int getLength() const;

        /* Returns true if the queue is empty. */
        bool isEmpty() const;

        /* Print out the PriorityQueue */
        void print() const;

        /* Pushes the val to the top of the queue. */
        bool push(const T &val);

        /* Removes and returns the top element from the queue. */
        T pop();

        /* Returns if the two lists contain the same elements in the
         * same order.
         */
        bool operator==(const PriorityQueue<T> &pq) const;
};

#include "priorityqueue.cpp"

#endif
\end{lstlisting}

\vskip.15in
\noindent\textbf{Write some test cases:} \\
Create some test cases, using cxxtestgen, that you believe would cover all aspects of your code.

\vskip.15in
\noindent\textbf{STL:} \\
You may use the STL, but not queue or priority\_queue.

\vskip.15in
\noindent\textbf{Memory Management:} \\
Now that are using new, we must ensure that there is a corresponding delete to free the memory.  Ensure there are no memory leaks in your code!  Please run Valgrind on your tests to ensure no memory leaks!
\vskip.15in

\vskip.15in
\noindent\textbf{How to turn in:} \\
Turn in via GitHub.  Ensure the file(s) are in your directory and then:
\begin{itemize}
\item \$ git add $<$files$>$
\item \$ git commit 
\item \$ git push
\end{itemize}

\vskip.15in
\noindent\textbf{Due Date:}
April 03, 2019 2359

\vskip.15in
\noindent\textbf{Teamwork:} No teamwork, your work must be your own.

\end{document}
