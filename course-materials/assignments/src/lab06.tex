\documentclass[12pt]{article}
\usepackage{listings}
\usepackage{color}
\textwidth=7in
\textheight=9.5in
\topmargin=-1in \headheight=0in
\headsep=.5in
\hoffset  -.85in

\definecolor{mygray}{rgb}{0.4,0.4,0.4}
\definecolor{mygreen}{rgb}{0,0.8,0.6}
\definecolor{myorange}{rgb}{1.0,0.4,0}

\lstset{
basicstyle = \ttfamily,columns=fullflexible,
commentstyle=\color{mygray},
frame=single,
numbers=left,
numbersep=5pt,
numberstyle=\tiny\color{mygray},
keywordstyle=\color{mygreen},
showspaces=false,
showstringspaces=false,
stringstyle=\color{myorange},
tabsize=2
}

\pagestyle{empty}

\renewcommand{\thefootnote}{\fnsymbol{footnote}}

\begin{document}

\begin{center}
{\bf lab 06: Review of C++ Arrays}

\end{center}

\setlength{\unitlength}{1in}

\begin{picture}(6,.1) 
\put(0,0) {\line(1,0){6.25}}         
\end{picture}

\renewcommand{\arraystretch}{2}
\setlength{\tabcolsep}{6pt} % General space between cols (6pt standard)
\renewcommand{\arraystretch}{.5} % General space between rows (1 standard)

\vskip.15in
\noindent\textbf{How to Add the Webhook:}
To add the webhook:\\
\begin{enumerate}
\item Open up \textbf{your} github on Github.com
\item Go to settings.
\item Click Webhooks.
\item Click add webhooks.
\item The Payload URL is: \\
http://csci.csuniv.edu:2234/github/build-csci-315-spring-2019.php \\
\textbf{\underline{Select json for content type.}}
\item Click add webhook.
\end{enumerate}
Remember, after the first push, please wait 5-10 minutes for the auto-grader to get your repository.  Then subsequent pushes should receive a grade.


\vskip.15in
\noindent\textbf{Instructions:} This lab is to review arrays in C++.  While Java arrays have similarities with C++ arrays, there are differences.  First, Java arrays are objects, while C++ arrays are just sequential list of elements.  The key usage difference is that C++ does not store any information about the array (like length.)  Therefore any extraneous information must be stored separately. \\

\textbf{Note:}Your header file should be in src/arrays.hpp, your C++ code should be in src/arrays.cpp \\

I do not need a main file.

Implement the following functions:
\begin{lstlisting}[language=C++]{Name=test2}
/* Returns the number of elements in the array that are less than 0. */
int countNegatives(const int array[], int size);

/* Find the smallest odd value in the array. 
 * Returns -1 if there is no odd value in the array.
 */
int findMinOdd(const int array[], int size);

/* Returns the index in the array where value is found.  
 * Return -1 if the value is not present in the array.
 */
int search(const int array[], int size, int value);

/* Removes an item at position index by shifting later elements left.
 * Returns true iff 0 <= index < size.
 */
bool remove(int array[], int size, int index);
\end{lstlisting}

\vskip.15in
\noindent\textbf{Testing:} \\
Please write some test cases to ensure you code is working correctly.  You may utilize CXXtestgen or write you own manual test cases in a main file.

\vskip.15in
\noindent\textbf{How to turn in:} \\
Turn in via GitHub.  Ensure the file(s) are in your directory and then:
\begin{itemize}
\item \$ git add $<$files$>$
\item \$ git commit 
\item \$ git push
\end{itemize}

\vskip.15in
\noindent\textbf{Due Date:}
February 04, 2019 2359

\vskip.15in
\noindent\textbf{Teamwork:} No teamwork, your work must be your own.

\end{document}
