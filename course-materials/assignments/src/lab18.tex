\documentclass[12pt]{article}
\usepackage{listings}
\usepackage{color}
\textwidth=7in
\textheight=9.5in
\topmargin=-1in
\headheight=0in
\headsep=.5in
\hoffset  -.85in

\definecolor{mygray}{rgb}{0.4,0.4,0.4}
\definecolor{mygreen}{rgb}{0,0.8,0.6}
\definecolor{myorange}{rgb}{1.0,0.4,0}

\lstset{
basicstyle = \ttfamily,columns=fullflexible,
commentstyle=\color{mygray},
frame=single,
numbers=left,
numbersep=5pt,
numberstyle=\tiny\color{mygray},
keywordstyle=\color{mygreen},
showspaces=false,
showstringspaces=false,
stringstyle=\color{myorange},
tabsize=2
}

\pagestyle{empty}

\renewcommand{\thefootnote}{\fnsymbol{footnote}}

\begin{document}

\begin{center}
{\bf lab 18 Hash Tables with Chaining}

\end{center}

\setlength{\unitlength}{1in}

\begin{picture}(6,.1) 
\put(0,0) {\line(1,0){6.25}}         
\end{picture}

\renewcommand{\arraystretch}{2}
\setlength{\tabcolsep}{6pt} % General space between cols (6pt standard)
\renewcommand{\arraystretch}{.5} % General space between rows (1 standard)

\vskip.15in
\noindent\textbf{Instructions:} This lab is a practice in constructing a Hash Table with chaining.  Implement a Hash Table whose constructor take an integer (the initial size of the hash table), insert, remove, and get.  Hints: if the value is not found in the Hash Table return a value using the default constructor.  Also, use your previous code!

\begin{lstlisting}[language=C++]{Name=test2}
#ifndef HASH_TABLE_H
#define HASH_TABLE_H

/* HashTable via chaining */
template<class K, class V>
class HashTable {
    private:
        /* Class to begin filling out...*/
    public:
        /* Initialize the Hash Table with size size. */
        HashTable(const int size);

        /* Deconstructor shall free up memory */
        ~HashTable();

        /* Map key -> val.
         * Return true if sucessful (it is unique.)
         * Otheriwise return false.
         */
        bool insert(const K &key, const V &val);

        /* Print out the HashTable */
        void print() const;

        /* Remove the val associated with key.
         * Return true if found and removed.
         * Otherwise return false.
         */
        bool remove(const K &key);

        /* Retrieves the V val that key maps to. */
        V& operator[](const K &key);
};

int hashcode(int key);
int hashcode(std::string &key);

#include "hashtable.cpp"

#endif
\end{lstlisting}

\vskip.15in
\noindent\textbf{Write some test cases:} \\
Create some test cases, using cxxtestgen, that you believe would cover all aspects of your code.

\vskip.15in
\noindent\textbf{Memory Management:} \\
Now that are using new, we must ensure that there is a corresponding delete to free the memory.  Ensure there are no memory leaks in your code!  Please run Valgrind on your tests to ensure no memory leaks!
\vskip.15in

\vskip.15in
\noindent\textbf{STL:} \\
You may use vector, queue/deque, and list from the STL.  Do not use any other data structure (especially map!)  Failure to follow these instructions is an automatic 0 for this lab.

\vskip.15in
\noindent\textbf{How to turn in:} \\
Turn in via GitHub.  Ensure the file(s) are in your directory and then:
\begin{itemize}
\item \$ git add $<$files$>$
\item \$ git commit 
\item \$ git push
\end{itemize}

\vskip.15in
\noindent\textbf{Due Date:}
March 27, 2019 2359

\vskip.15in
\noindent\textbf{Teamwork:} No teamwork, your work must be your own.

\end{document}
