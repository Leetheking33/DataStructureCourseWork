\documentclass[12pt]{article}
\usepackage{listings}
\usepackage{color}
\textwidth=7in
\textheight=9.5in
\topmargin=-1in
\headheight=0in
\headsep=.5in
\hoffset  -.85in

\definecolor{mygray}{rgb}{0.4,0.4,0.4}
\definecolor{mygreen}{rgb}{0,0.8,0.6}
\definecolor{myorange}{rgb}{1.0,0.4,0}

\lstset{
basicstyle = \ttfamily,columns=fullflexible,
commentstyle=\color{mygray},
frame=single,
numbers=left,
numbersep=5pt,
numberstyle=\tiny\color{mygray},
keywordstyle=\color{mygreen},
showspaces=false,
showstringspaces=false,
stringstyle=\color{myorange},
tabsize=2
}

\pagestyle{empty}

\renewcommand{\thefootnote}{\fnsymbol{footnote}}

\begin{document}

\begin{center}
{\bf lab 22 Graphs via Adjacency Matrix}

\end{center}

\setlength{\unitlength}{1in}

\begin{picture}(6,.1) 
\put(0,0) {\line(1,0){6.25}}         
\end{picture}

\renewcommand{\arraystretch}{2}
\setlength{\tabcolsep}{6pt} % General space between cols (6pt standard)
\renewcommand{\arraystretch}{.5} % General space between rows (1 standard)

\vskip.15in
\noindent\textbf{Instructions:} In this lab implement a Graph with an adjacency matrix.

Implement the following class:
\begin{lstlisting}[language=C++]{Name=test2}
#ifndef GRAPHAM_H
#define GRAPHAM_H

/* This class represents a weighted directed graph via an adjacency matrix.
 * Vertices are given an index, starting from 0 and ascending
 * Class W : W represent the weight that can be associated with an edge.
 * We will not weight the vertices.
 * W is the data type for the weight.  Normally an int.
 */

template<class W>
class GraphAM {
    private:
        /* Recommended, but not necessary. */
        void depthFirstTraversal(void (*visit)(const int node),
                int *visited, const int cVertex);
        /* You fill out private member variables. */
    public:
        /* Initialize an empty graph. */
        GraphAM();

        /* Initialize the Graph with a fixed number of vertices. */
        GraphAM(const int vertices);

        /* Deconstructor shall free up memory */
        ~GraphAM();

        /* Removes a vertex. 
         * return whether successful or not
         * Note: You must shift all vertices accordingly.
         */
        bool removeVertex(int idx);

        /* Adds amt vertices to the graph. Returns the starting point 
         * of the vertice count.
         */
        int addVertices(int amt);

        /* Adds an edge with weight W to the graph.
         * The return is for you convience and will not be graded.
         */
        bool addEdge(const int start, const int end, const W &weight);

        /* 
         * Remove edge from graph.
         * The return is for you convience and will not be graded.
         */
        bool removeEdge(const int start, const int end);

        void depthFirstTraversal(void (*visit)(const int node));
        void breadthFirstTraversal(void (*visit)(const int node));

        /*
         * Return adjacent weight from start to end (or -1 if they are
         * not adjacent.
         */
        W adjacent(const int start, const int end);

        /* Run Dijkstra's Shortest Path to find the shortest path from start
         * to end and returning that smallest weight.
         * return -1 if a path does not exist!
         */
        W dijkstraShortestPath(const int start, const int end);

        /* Print out the Graph */
        void print() const;

};

#include "grapham.cpp"

#endif
\end{lstlisting}

\vskip.15in
\noindent\textbf{Write some test cases:} \\
Create some test cases, using cxxtestgen, that you believe would cover all aspects of your code.

\vskip.15in
\noindent\textbf{Memory Management:} \\
Now that are using new, we must ensure that there is a corresponding delete to free the memory.  Ensure there are no memory leaks in your code!  Please run Valgrind on your tests to ensure no memory leaks!
\vskip.15in

\vskip.15in
\noindent\textbf{How to turn in:} \\
Turn in via GitHub.  Ensure the file(s) are in your directory and then:
\begin{itemize}
\item \$ git add $<$files$>$
\item \$ git commit 
\item \$ git push
\end{itemize}

\vskip.15in
\noindent\textbf{Due Date:}
April 15, 2019 2359

\vskip.15in
\noindent\textbf{Teamwork:} No teamwork, your work must be your own.

\end{document}
